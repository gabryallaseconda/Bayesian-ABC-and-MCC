\documentclass[9pt]{beamer}
\usepackage[utf8]{inputenc}
\usepackage[T1]{fontenc}
\usepackage{lmodern}
\usetheme{Hannover}
\begin{document}
	\author{Gabrielli - Di Filippo - Caldarini - Musiari - Bertoni}
	\title{Coupled Markov chains with applications to Approximate Bayesian Computation for model based clustering}
	%\subtitle{}
	%\logo{}
	%\institute{}
	%\date{}
	%\subject{}
	%\setbeamercovered{transparent}
	%\setbeamertemplate{navigation symbols}{}
	\begin{frame}[plain]
		\maketitle
	\end{frame}
	
	\begin{frame}
		\frametitle{A complex problem}
		
		\textit{}
		
	\end{frame}

	\begin{frame}
		Aggiungere stella 1
		
		Main obstacle to parallelization: in order to use multi-processor computation 
		The idea is to split up the Markov chain into many smaller chains which execution can be parallelized.
		The use of a single long Markov chain is required for convergence as the estimator used is biased. In order to have convergence with shorter chains we need an unbised estimator.
		
		The main advantages to use an unbiased estimator are:
		\begin{itemize}
			\item
		\end{itemize}
		         
	\end{frame}
	
	\begin{frame}
		\frametitle{Assumptions}
		\begin{itemize}
			\item
			\item
			\item
			\item
		\end{itemize}
	\end{frame}
\end{document}