\documentclass[9pt]{beamer}
\usepackage[utf8]{inputenc}
\usepackage[T1]{fontenc}
\usepackage{lmodern}
\usetheme{Hannover}


%%%%%%%%%%%%%%%%%%%%%%%%%%%%%%%%%%%%%%%%%%%%%
%STRUCTURE OF THE PRESENTATIONS
% TITLE
% INTRODUCTION
% -	Problems
% APPROXIMATE BAYESIAN COMPUTATION
% ......
% UNBIASED MARKOV CHAIN MONTE CARLO






\begin{document}
	\author{Gabrielli - Di Filippo - Caldarini - Musiari - Bertoni}
	\title{Coupled Markov chains with applications to Approximate Bayesian Computation for model based clustering}
	%\subtitle{}
	%\logo{}
	%\institute{}
	%\date{}
	%\subject{}
	%\setbeamercovered{transparent}
	%\setbeamertemplate{navigation symbols}{}
	\begin{frame}[plain]
		\maketitle
	\end{frame}
	
	\begin{frame}
		\frametitle{A complex problem}
		
		In order to speed up computation time, P. Jacob, J. O'Leary and Y. Atchadé proposed in 2020 a way parallelize Monte Carlo algorithm using Markov chain couplings.
		
	\end{frame}

	\begin{frame}
		\frametitle{The road to parallelization}
		
		Standard Markov chain Monte Carlo methods are potentially biased for any fixed numbers of iterations.\\
		
		The idea is to slit up the long Markov chain into many shorter coupled Markov chains; to allow this we need to start from an unbiased MCMC method.\\
		
		P. Glynn and C. Rhee proposed in 2018 the class of exact estimations algorithms.\\
		
		
	\end{frame}
	
	\begin{frame}
		\frametitle{What is Markov chains coupling?}
		
		Markov chain coupling allow to reduce the convergence time.
		
		(Completare con una spiegazione delle note)
	\end{frame}
		

	
	\begin{frame}
		\frametitle{Rhee-Glynn estimator}
		The goal is to estimate
		$$
			\mathbb{E}_{\pi}[h(X)] 
			= \int h(x) \pi (\text{d}x)
		.
		$$
		The estimator we are going to construct is based on a coupled pair of Markov chains, $(X_t)_{t\geq 0}$ and $(Y_t)_{t\geq}$, which marginally start from $\pi_0$ and evolve accordingly to $P$.
		
		We consider some assumptions:
		\begin{enumerate}
			\item As $t \to \infty$, 
			$$ \mathbb E [h(X_t)] \to \mathbb E_\pi [h(X)];$$
			and there exists $\eta > 0$ and $D < \infty$ such that $\mathbb E [|h(X_t)|^{2 + \eta}] \leq D$ for all $t \geq 0$. 
			\item
			\item
		\end{enumerate}

	\end{frame}
	\begin{frame}
			\frametitle{Rhee-Glynn estimator}
	Thanks to the previous assumptions we can prove that:
	\begin{align*}
		\mathbb{E}_{\pi}[h(X)] 
		& = \mathbb{E}[h(X_k)] + \sum_{t = k+1}^\infty\{\mathbb{E}[h(X_t)] - \mathbb{E}[h(X_{t-1})]\} \\
		& \cdots\\
		&  = \mathbb{E}[h(X_k) + \sum_{t = k+1}^{\tau -1}\{h(X_t) - h(Y_{t-1})\} ]
	;
	\end{align*}
	
	The Rhee-Glynn estimator is:
	$$ ...
	$$
	which is unbiased by construction.
	
	\end{frame}


	\begin{frame}
		\frametitle{Time-averaged estimator}
		The Rhee-Glynn has to be make computationally feasible, the time-averaged estimator keeps the assumptions allowing the computation.
		$$
		H_{k:m}
		= \underbrace{
			\frac{1}{m-k+1}\sum_{l=k}^{m}h(X_l) }_{MCMC_{k:m}}
		+ \underbrace{ 
			\sum_{l=k+1}^{\tau -1}\min(1, \frac{l-k}{m-k+1})\{h(X_l)-h(Y_{l-1})\} }_{BC_{k:m}}
		$$ %m-k deve essere comunque grande, m>> k
		\begin{itemize}
			\item $MCMC_{k:m}$  is the standard MCMC average
			\item $BC_{k:m}$ is the bias correction
		\end{itemize}
		\begin{itemize}
			\item $k-1$ number of burn-in iterations;
			\item $m$ is a fixed integer of number of maximum iterations,
			\item $\tau$ is a random variable representing the meeting time:
			$$ \tau = \inf\{t \geq 1 : X_t = Y_{t-1}\}$$
		\end{itemize}


	
	\end{frame}
\begin{frame}
	\frametitle{Application to Metropolis-Hasting algorithm}
	Both GS and MH satisfy the assumptions. Here how to adapt MH with coupling:
	
	\begin{itemize}
		\item[Step 1:] Draw $X_0$ and $Y_0$ from an initial distribution $\pi_0$ and draw $X_1 \sim P(X_0, \cdot)$;
		\item[Step 2:]Set $t=1$: while $t<\max\{m,\tau\}$ and:
		\begin{itemize}
			\item[a] draw $(X_{t+1}, Y_t)\sim \bar P \{(X_t, Y_{t-1}), \cdot \}$;
			\item[b] set $t \leftarrow t+1$;
		\end{itemize}
		\item[Step 3:] compute 
		$$ H_{k:m}(X,Y)$$
		using the previous formula.
%		for each $l \in \{l, \dots, m\}$, compute
%		$$
%		H_l(X,Y) = h(X_l) + \sum^{\tau -1}_{t= l + 1} \{
%		h(X_t) - h(Y_{t-1})\}
%		;
%		$$
%		return 
%		$$
%		H_{k:m}(X,Y)
%		=(m-k+1)^{-1} \sum ^{m}_l=k H_l(X,Y)$$
%		or compute 
%		$$ H_{k:m}(X,Y) 
%		= 
%			\frac{1}{m-k+1}\sum_{l=k}^{m}h(X_l) 
%		+  
%			\sum_{l=k+1}^{\tau -1}\min(1, \frac{l-k}{m-k+1})\{h(X_l)-h(Y_{l-1})\} 
%		$$
	\end{itemize}
	
	The following is the algorithm to calculate the coupled kernel $\bar P \{(X_t, Y_{t-1}), \cdot \}$ via MH:
	
	
	\begin{itemize}
		\item[Step 1:] Sample $(X^\star, Y^\star) | (X_t, Y_{t-1})$ from a maximal coupling of $q(X_t, \cdot)$ and $q(Y_{t-1}, \cdot)$;
		\item[Step 2:] sample $U \sim \mathcal{U}([0,1])$;
		\item[Step 3:] if
		$$ U
		\leq \min\bigg \{
		1,
		\frac{ \pi(X^\star)q(X^\star,X_t)}{
		\pi(X_t)q(X_t, X^\star)}
		\bigg \}
		$$
		then $X_{t+1} = X^\star$; otherwise $X_t = X_{t-1}$.
		\item[Step 4:] if
		$$ U
		\leq \min\bigg \{ 
		1,
		\frac{ \pi(Y^\star)q(Y^\star,Y_t)}{
			\pi(Y_t)q(Y_t, Y^\star)}
		\bigg \}
		$$
		then $Y_{t+1} = Y^\star$; otherwise $Y_t = Y_{t-1}$.
		
	\end{itemize}
\end{frame}
\end{document}